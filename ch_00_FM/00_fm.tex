
\frontmatter

\title{Programming the Pico}
\subtitle{Learn Coding and Electronics with the Raspberry Pi Pico}
\date{\vspace{-5ex}}
\author{Simon Monk} 

\begin{copyright}
Copyright Simon Monk 2023

Tutti i diritti riservati.

Seconda Edizione

\vspace{0.5in}

Nessuna parte di questo libro può essere riprodotta o trasmessa in qualsiasi forma o con qualsiasi mezzo, elettronico o meccanico, comprese fotocopie, registrazioni o qualsiasi altro sistema di archiviazione e recupero delle informazioni, senza il permesso scritto dell'editore.

Limite di responsabilità/Esclusione di garanzia: sebbene l'editore e l'autore abbiano fatto del loro meglio per preparare questo libro, non rilasciano dichiarazioni o garanzie rispetto all'accuratezza o alla completezza dei contenuti di questo libro e specificatamente declinano qualsiasi garanzia implicita di commerciabilità. o idoneità per uno scopo particolare. Nessuna garanzia può essere creata o estesa da rappresentanti di vendita o materiale di vendita scritto. I consigli e le strategie contenuti nel presente documento potrebbero non essere adatti alla propria situazione. In caso di dubbi, consultare un professionista, se necessario. Né l'editore né l'autore saranno responsabili per eventuali perdite di profitto o altri danni commerciali, inclusi ma non limitati a danni speciali, incidentali, consequenziali o di altro tipo.

\vspace{1in}

\begin{figure}[hbt!]
\includegraphics[width=4cm]{ch_00_FM/logo_mm_press.png}
\end{figure}

ISBN: 978-1-7394874-3-0

\end{copyright}

\maketitle


\section*{Prefazione alla seconda edizione}

La Raspberry Pi Foundation ha ottenuto un enorme successo con la sua gamma di computer a scheda singola. Dal rilascio del Raspberry Pi originale nel 2014, il Raspberry Pi si è evoluto in una macchina che costituisce un sostituto perfettamente rispettabile di un computer desktop più tradizionale. Inoltre, le versioni Raspberry Pi da zero a 5 e Pi 400 hanno sempre fornito l'accesso ai pin GPIO per consentire l'interfacciamento diretto di sensori, display e tutti i tipi di elettronica con il Raspberry Pi.

Il Raspberry Pi Pico e il suo compagno esteso, il Pico W, rappresentano una svolta radicale rispetto a tutte le versioni precedenti del Raspberry Pi. A differenza di altri Raspberry Pi, Pico e Pico W non hanno un sistema operativo né interfacce integrate per tastiera, mouse e monitor: piuttosto, sono rivolti direttamente agli appassionati di elettronica e agli educatori che desiderano conoscere l'informatica fisica.

Il Raspberry Pi Pico e Pico W (chiamiamoli semplicemente Pico) non sono computer normali, sono schede a microcontrollore. In altri termini, non sono intesi per l'informatica generica, ma piuttosto sono progettati per aiutare a realizzare progetti elettronici - praticamente per essere il cervello di tali progetti. Ad esempio, un Pico potrebbe essere attivato per realizzare un robot, controllare motori e un altoparlante per emettere suoni, oppure potrebbe essere utilizzato per visualizzare la temperatura o le letture di altri sensori su un piccolo schermo LCD.

Il Raspberry Pi Pico W funziona proprio come un Raspberry Pi Pico, ma aggiunge la possibilità di utilizzare il WiFi, rendendolo un'ottima scheda per progetti collegati alla Internet of Things.

La scheda microcontrollore standard di riferimento, ampiamente utilizzata nell'istruzione e dagli hobbisti, è l'Arduino Uno R3. Il Pico è saldamente nello stesso territorio dell'Uno, ma ha un design molto più aggiornato e potente. Inoltre, il suo prezzo è estremamente competitivo ed è addirittura apparso come omaggio sulla copertina della rivista Hackspace. Mentre un Arduino è programmato nel linguaggio di programmazione snello ed efficiente C++ (che può essere utilizzato anche sul Pico), il processore ARM dual-core del Pico è in grado di eseguire il linguaggio Python, più affamato di memoria, ma è un linguaggio di programmazione più popolare nel mondo e ampiamente utilizzato dagli educatori. Per la maggior parte delle persone è più semplice iniziare a programmare con Python che con C++ e l'implementazione ufficiale di Python consigliata dalla Raspberry Pi Foundation (MicroPython) risulta sufficientemente veloce per la maggior parte dei progetti.


Questo libro insegna il Python e allo stesso tempo si impara come utilizzare Pico. Non si presuppone né è richiesta alcuna conoscenza precedente di programmazione o elettronica per imparare Python e far fare al tuo Pico cose davvero interessanti.
Quando si tratta di esplorare il lato hardware delle cose, avrai bisogno di un ordinare alcuni componenti elettronici per sfruttare al massimo il tuo Pico.

Acquistare le parti di cui hai bisogno può essere complicato se sei nuovo nel campo dell'elettronica e quindi, in questo libro, utilizziamo il kit breadboard MonkMakes per Raspberry Pi Pico. Questo kit è stato progettato appositamente per questo libro e include una buona gamma di componenti di base per iniziare.

Se preferisci non lasciarti coinvolgere troppo dall'elettronica, puoi semplicemente utilizzare Pico come veicolo per imparare Python utilizzando il LED integrato di Pico e un piccolo spezzone di cavo sarà sufficiente per provare un'intera gamma di progetti.

La prima edizione di questo libro si concentrava sul Pico. In questa seconda edizione il cambiamento principale è un nuovo capitolo che spiega come utilizzare le funzionalità WiFi del Pico W.

\section*{Come usare questo libro}
Imparare a programmare, come qualsiasi altra abilità, richiede pratica e esempi semplici e diretti da seguire. In questo libro verrai guidato passo dopo passo attraverso programmi di esempio, intervallati da spiegazioni e background.

Il primo capitolo è un'introduzione al Pico, che fornisce una visita guidata alle schede Pico e Pico W e spiega qualcosa in più sulle loro caratteristiche e su cosa puoi usarle. Questo dovrebbe farti venire voglia di usare davvero il tuo Pico, quindi il Capitolo 2 è una guida per iniziare: caricherai il tuo primo programma sul tuo Pico ed eseguirai alcuni esperimenti, solo per abituarti all'uso del dispositivo.

I capitoli da 3 a 6 riguardano interamente Python e sono illustrati con un ampio esempio di codice Morse che utilizza il LED integrato di Pico per far lampeggiare i messaggi. Utilizzando il LED integrato non è necessario utilizzare altri componenti; l'esempio diventa gradualmente più sofisticato man mano che la tua conoscenza di Python cresce.

I capitoli da 7 a 10 mostrano come utilizzare MicroPython per utilizzare sensori e pulsanti elettronici e come interfacciarsi con i servomotori.

Il capitolo 12 tratta delle funzionalità WiFi del Pico W e mostra come connettersi a Internet e gestire pagine web.

Questo libro è incentrato sul MicroPython, ufficialmente raccomandato; tuttavia, MicroPython ha un concorrente in CircuitPython di Adafruit. Nel capitolo 13 imparerai le differenze tra MicroPython e CircuitPython, oltre a sapere dove trovare ulteriori risorse di CircuitPython.

Avendo usato questo libro per imparare Python, non ti sarà difficile provare l'alternativa CircuitPython. Questo capitolo fornisce anche una breve introduzione alla programmazione del tuo Pico in C++, utilizzando il software Arduino, nel caso volessi provarlo.

\subsection*{Scaricamento del codice}
Tutti i sorgenti di esempio per questo libro sono disponibili per il download attraverso GitHub al seguente indirizzo:

\url{https://github.com/simonmonk/prog_pico_ed2}

Il modo più semplice per averli sul computer consiste nello scaricare il file ZIP all'indirizzo:

\url{https://github.com/simonmonk/prog_pico_ed2/archive/main.zip}

ed estrarre il contenuto in una cartella a scelta.

\subsection*{Componenti elettroniche ed Hardware}
Puoi trovare informazioni sul kit breadboard MonkMakes per Raspberry Pi
Pico su \url{https://www.monkmakes.com/pico_kit1} dove troverai un elenco di fornitori in tutto il mondo da cui puoi acquistare il kit.

\subsection*{Il website del libro}
Per gli errata, e ogni ulteriore informazione sul libro,visita il sito weball'indirizzo: 

\url{http://simonmonk.org/pico_book_2}


\section*{Ringraziamenti}


Mille grazie a Ian Huntley e Mike Basset per la loro utile revisione tecnica e il copy editing della prima edizione. Grazie anche a David Whale, Dave Sanderson e Brett (@Brett0123456) per aver dedicato del tempo a rintracciare gli errata della prima edizione, che ora sono stati corretti.

I layout della breadboard per questo libro sono stati creati utilizzando l'eccellente software Fritzing (fritzing.org). Altri diagrammi sono stati disegnati utilizzando Inkscape (inkscape.org). Grazie anche ai creatori di Thonny per aver creato un editor di codice per principianti così eccezionale per Python.


\tableofcontents
